\documentclass[a4paper,english,numbers=noenddot,bibliography=totoc,chapterprefix=on,DIV=12]{scrartcl}

\usepackage[utf8x]{inputenc}
\usepackage{graphicx}
\usepackage{color}
\usepackage{amssymb,amsmath}
\usepackage{enumitem}
\usepackage{bookmark,hyperref}
\usepackage{multirow}
\usepackage[labelfont=bf,font=small]{caption}
\usepackage[font=footnotesize]{subcaption}
\usepackage{rotating}

\usepackage{subcaption}
\usepackage{tabularx}

\setcapindent{0mm}

% boldmath in headings and toc, but not headers
\def\bfseries{\fontseries \bfdefault \selectfont \boldmath}

%allows footnotes in tabular
\usepackage{footnote}
\makesavenoteenv{tabular}

% avoid space issues after macro
\usepackage{xspace}
\newcommand{\tof}{TOF\xspace}



\begin{document}

\title{Questionnaire: Requirements on Live Data Reduction and Experiment Control Software}
\author{Simon Heybrock, Jonathan Taylor, Michael Wedel\\
    {\small\href{mailto:simon.heybrock@esss.se}{\nolinkurl{simon.heybrock@esss.se}}},{\small\href{mailto:jonathan.taylor@esss.se}{\nolinkurl{jonathan.taylor@esss.se}}},{\small\href{mailto:michael.wedel@esss.se}{\nolinkurl{michael.wedel@esss.se}}}}

\maketitle

\tableofcontents


\section{Motivation}


\section{Questions}

\begin{itemize}
  \item User sits down in front of computer to run their experiment.
    What do they need to do, typically?
    What are common work flows?
  \item What are typical time-scales?
    \begin{itemize}
      \item For (visual) feedback (e.g., an intensity plot)?
      \item For manual control?
      \item For automated control?
    \end{itemize}
  \item What do users need to see during the experiment?
    \begin{itemize}
      \item ``What is the detector seeing now?''
      \item Integral over the full length of the run?
      \item Compare data from different parameters regions of the run (e.g., different temperatures)?
    \end{itemize}
  \item What kind of interaction with the live data reduction is required \emph{during a run}?
    \begin{itemize}
      \item None.
      \item Simple (re-bin, create plots for data from earlier in the run).
      \item Create/modify reduction script while running.
    \end{itemize}
  \item What would event mode be used for?
    \begin{itemize}
      \item Not crucial during experiment.
      \item Filtering (on what aspects?).
      \item Re-binning.
    \end{itemize}
  \item What are typical time periods for gathering or accumulating data?
    \begin{itemize}
      \item With fixed experiment conditions?
      \item With varying conditions (e.g., a temperature ramp)?
    \end{itemize}
  \item Monitors:
    \begin{itemize}
      \item What kinds of monitors will the instrument have?
      \item Will monitors run in event mode?
        What event rates are to be expected?
      \item If yes, is this required in the reduction (for what reasons)?
      \item Can we circumvent event mode by histogramming the monitors on, e.g., a per-pulse basis?
    \end{itemize}
  \item Is there anything that you have to or want to do with the 3D detectors?
    \begin{itemize}
      \item No.
      \item Unclear.
      \item Find trajectories to remove scattering by anything but the sample.
    \end{itemize}
  \item What needs to be stored during a live reduction?
    \begin{itemize}
      \item Plots, e.g., $I(Q)$ plot?
      \item Big ``reduced'' data (i.e., before integrating, slicing, and plotting)? If not, will users typically re-reduce things after the experiment or are most of them happy with the plots obtained during the experiment?
    \end{itemize}
  \item For the questions above, are there big differences between:
    \begin{itemize}
      \item Normal users?
      \item Power users?
      \item Instrument scientist?
    \end{itemize}
\end{itemize}



\end{document}
